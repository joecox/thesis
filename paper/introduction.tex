\chapter{Introduction}

By designing a static information flow analysis to accurately model
the dynamic taint tracking performed by \phosphor{}, a significant
reduction in the instrumentation required by \phosphor{} is achieved.

Dynamic taint tracking, also called dynamic information flow analysis,
is the technique of automatically tracking the paths of certain pieces
of data through a program during runtime. As such, information about
the data can be discovered as it enters and exits the program, is
written to and read from, and affects other parts of the
program. Taint tracking is useful for both determining the origin of a
piece of data as well as knowing information about its state during
program execution. Data is tagged as it enters the program through
user input or some other method, and tags are propagated as data is
operated on or combined with other data. The method of tag propagation
may vary but must be done in a way that the origins of a piece of data
are discoverable by inspecting the tag.

Plenty of motivation exists for taint tracking, including the
canonical example of detecting injection attacks such as SQL
injections. Tracking data such as user input can prevent injection of
potentially malicious data into critical functions or SQL command
processing methods, which is the focus of some previous works
\cite{sql1} \cite{sql2}. Many solutions and systems exist, all with
various limitations. Some systems \cite{stringtaint} \cite{sql1} modify
programs at the bytecode level but only track data flow of String-type
objects. Another \cite{jikes} can track data flow of all Java types
but only targets an incomplete ``research JVM'' (read: not Oracle or
OpenJDK).  Most taint tracking systems suffer from bad performance,
soundness, precision, portability, or some combination of the four.

\phosphor{} \cite{phosphor_oopsla} is a taint tracking system for the
Java Virtual Machine (JVM). It differs fundamentally from prior work
in how it stores and propagates taint tags and is able to be
performant, sound, precise, and portable. \phosphor{} does not require
any JVM modification and achieves taint tracking through instrumenting
program bytecode. Unlike previous approaches, \phosphor{} stores taint
tags as shadow variables, one for each concrete variable in the
program. Runtime operations cause modification and combination of
taint tags. For example, the resulting taint tag of an arithmetic
operation is the bitwise OR of the taint tags of the operands. Aside
from its portability, soundness, and precision achievements, \phosphor{}
introduces, on average, a 53\% performance overhead. \phosphor{}'s main
mode of usage is to provide source and sink files, which enumerate the
methods from which data of interest originates and methods into which
we want to know if tainted data enters.

For simplicity and completeness, \phosphor{} instruments the entire
application bytecode and assumes the presence of an instrumented
JVM. Previous work \cite{manoj_project}, on which this paper's author
was a participant, explores only partially instrumenting the
application bytecode, doing so with a new tool called \phosphorpi{}. The
key insight of this work is that that only a subset of the methods in
the program callgraph, and methods that read or write shared data with
those methods, need to be instrumented. Except for some special cases,
methods that do not lie on any of the paths from any source to any
sink are of no interest and can remain uninstrumented. This prior work
involved producing \phosphor{}pi by modifying \phosphor{} to support
partial instrumentation and performing static analysis to generate the
list of methods to be instrumented. This approach is able to save, on
average, 11\% of performance overhead compared to \phosphor{}.

However, this prior work has certain limitations. First, it fails to
consider some program behaviors, such as calls into native
libraries. Second, it is imprecise due to a number of special cases
which result in more instrumented code. In this paper, we describe a
different taint analysis for partial instrumentation that more
completely and accurately models the taint tracking done by
\phosphor{}, including calls to native libraries. Our approach also
further reduces instrumentation. While \phosphor{} and \phosphorpi{}
store taint tags as 32-bit integers, we reduce them to boolean
properties, which allows us to remove instrumentation on certain
methods in which the taintedness of data is unchanged throughout the
method. The analysis described in this paper is designed to integrate
with further modifications made to \phosphor{} by its original author.

The remainder of the thesis is as follows. Chapter 2 presents
background information on relevant tools and prior work. Chapter 3
presents the high-level approach taken to achieve partial
instrumentation. Chapter 4 describes the implementation details behind
achieving partial instrumentation. Chapter 5 shares results of the
implemented approach. Chapter 6 concludes the thesis.
