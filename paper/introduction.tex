\chapter{Introduction}

Dynamic taint tracking, also called dynamic information flow analysis, is the technique of automatically tracking the paths of certain pieces of data through a program during runtime. As such, information about the data can be discovered as it enters and exits the program, is written to and read from, and affects other parts of the program. Taint tracking is useful for both determining the origin of a piece of data as well as knowing information about its state during program execution. Data is tagged as it enters the program through user input or some other method, and tags are propagated as data is operated on or combined with other data. The method of tag propagation may vary but must be done in a way that the origins of a piece of data are discoverable by inspecting the tag.

Plenty of motivation exists for taint tracking, including the canonical example of detecting injection attacks such as SQL injections. Tracking data such as user input can prevent injection of potentially malicious data into critical functions or SQL command processing methods \cite{sql1} \cite{sql2}. Many solutions and systems exist, all with varous limitations. \cite{stringtaint} and \cite{sql1} both modify programs at the bytecode level but only track data flow of String-type objects. \cite{jikes} can track data flow of all Java types but only targets an incomplete ``research JVM'' (read: not Oracle or OpenJDK).  Most taint tracking systems suffer from bad performance, soundness, precision, portability, or some combination of the four. 

Phosphor \cite{phosphor_oopsla} is a taint tracking system for the Java Virtual Machine (JVM). It differs fundamentally from prior work in how it stores and propagates taint tags and is able to be performant, sound, precise, and portable. Phosphor does not require any JVM modification and achieves taint tracking through instrumenting program bytecode. Unlike previous approaches, Phosphor stores taint tags as shadow variables, one for each concrete variable in the program. Runtime operations cause modification and combination of taint tags. For example, the resulting taint tag of an arithmetic operation is the bitwise OR of the taint tags of the operands. Aside from its portability, soundness, and precision achievements, Phosphor introduces, on average, a 53\% performance overhead. Phosphor's main mode of usage is to provide source and sink files, which enumerate the methods from which data of interest originates and methods into which we want to know if tainted data enters. 

For simplicity and completeness, Phosphor instruments the entire application bytecode and assumes the presence of an instrumented JVM. The key insight of prior work done in \cite{manoj_project} is that only a subset of the methods in the program callgraph need to be instrumented. Methods that do not lie on any of the paths from any source to any sink are of no interest and can remain uninstrumented. This work involved modifying Phosphor to support partial instrumentation and performing static analysis to generate the list of methods to be instrumented. This approach is able to save, on average, 11\% of performance overhead compared to Phosphor. 

However, this prior work fails to consider some program behaviors, such as calls into native libraries. We describe a different taint analysis for partial instrumentation that more completely and accurately models the taint tracking done by Phosphor, including calls to native libraries. Our approach also further reduces instrumentation. By making a key assumption about taint tags we are able to remove instrumentation on certain methods even if we know that tainted data passes through them. The analysis described in this paper is designed to integrate with further modifications made to Phosphor by its original author.

The remainder of the thesis is as follows. Chapter 2 presents background information on relevant tools and prior work. Chapter 3 presents the high-level approach taken to achieve partial instrumentation. Chapter 4 describes the implementation details behind achieving partial instrumentation. Chapter 5 shares results of the implemented approach. Chapter 6 concludes the thesis.
