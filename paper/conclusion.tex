\chapter{Conclusion}
We were able to write an improved information flow analysis, which
compared favorably to previous work. It enabled us to reduce the
amount of instrumentation required to track data with Phosphor by an
average of 79.9\%. This improvement came with a trade-off in
functionality, in which we restricted Phosphor to only track taint as
a boolean property instead of with a 32-bit integer as it originally
did.  We believe the trade-off is worthwhile given the potentially
significant reduction in performance overhead. 

We were not successful in completely evaluating our approach, as we
were not able to measure performance overhead by directly running the
updated version of Phosphor on the Dacapo benchmarks. Also, our
analysis was not alone in contributing to the reduction in
instrumentation. The built-in reachability analysis performed by
Petablox alone produced a 48.8\% reduction in instrumentation.
However, our analysis also made a significant contribution, being
responsible for about 31\% out of the total 79.9\% of the overall
reduction. We think that there is great potential for future work in
dynamic taint analysis by partial instrumentation, both to further
reduce performance overhead and to relax the taint tracking
restriction we placed on Phosphor without sacrificing performance.
