\abstract {
Dynamic taint tracking is an important field of study with many tools
and systems created to implement it in Java, including \phosphor{}, a
general purpose taint tracking tool designed for commodity JVMs like
Oracle and OpenJDK. \phosphor{} works by instrumenting core Java
libraries and the entire application bytecode with operations to
accurately propagate taint information. Prior work aimed at reducing
the performance overhead of \phosphor{} by doing partial
instrumentation. The analysis that determined which parts of the
program to instrument was effective but flawed.

This paper aims to improve that analysis and further reduce the
performance overhead by instrumenting less of the program. We use the
Petablox program analysis tool and custom Datalog rules to perform an
information flow analysis that better models \phosphor{}'s behavior,
including calls across native library boundaries. We find that we
obtain a reduction in the amount of a program that needs to be
instrumented by 79.9\%.  }
